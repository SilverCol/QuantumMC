\documentclass[a4paper]{article}
\usepackage[slovene]{babel}
\usepackage[utf8]{inputenc}
\usepackage[T1]{fontenc}
%\usepackage[margin=2cm, bottom=3cm, foot=1.5cm]{geometry}
\usepackage{float}
\usepackage{graphicx}
\usepackage{amsmath}
\usepackage{amssymb}
\usepackage{subcaption}
\usepackage{hyperref}

\newcommand{\tht}{\theta}
\newcommand{\Tht}{\Theta}
\newcommand{\dlt}{\delta}
\newcommand{\eps}{\epsilon}
\newcommand{\thalf}{\frac{3}{2}}
\newcommand{\ddx}[1]{\frac{d^2#1}{dx^2}}
\newcommand{\ddr}[2]{\frac{\partial^2#1}{\partial#2^2}}
\newcommand{\mddr}[3]{\frac{\partial^2#1}{\partial#2\partial#3}}

\newcommand{\der}[2]{\frac{d#1}{d#2}}
\newcommand{\pder}[2]{\frac{\partial#1}{\partial#2}}
\newcommand{\half}{\frac{1}{2}}
\newcommand{\forth}{\frac{1}{4}}
\newcommand{\q}{\underline{q}}
\newcommand{\p}{\underline{p}}
\newcommand{\x}{\underline{x}}
\newcommand{\liu}{\hat{\mathcal{L}}}
\newcommand{\bigO}[1]{\mathcal{O}\left( #1 \right)}
\newcommand{\pauli}{\mathbf{\sigma}}
\newcommand{\bra}[1]{\langle#1|}
\newcommand{\ket}[1]{|#1\rangle}
\newcommand{\id}[1]{\mathbf{1}_{2^{#1}}}
\newcommand{\tinv}{\frac{1}{\tau}}
\newcommand{\s}{\sigma}
\newcommand{\us}{\underline{\s}}
\newcommand{\vs}{\vec{\s}}
\newcommand{\vr}{\vec{r}}
\newcommand{\vq}{\vec{q}}
\newcommand{\vv}{\vec{v}}
\newcommand{\vo}{\vec{\omega}}
\newcommand{\uvs}{\underline{\vs}}
\newcommand{\expected}[1]{\left\langle #1 \right\rangle}
\newcommand{\D}{\Delta}

\begin{document}

    \title{\sc\large Višje računske metode\\
		\bigskip
		\bf\Large Kvantni Monte-Carlo}
	\author{Mitja Vodnik, 28182041}
	\date{\today}
	\maketitle

    Z metodo Monte-Carlo obravnavamo primer 1D kvantnega oscilatorja v potencialu $V(q) = \half q^2 + \lambda q^4$.
    Zanimata nas harmonski ($\lambda = 0$) in anharmonski ($\lambda = 1$) primer. \\

    Simulacijo izvajamo v zvezni pozicijski bazi $\{ \ket{q_j} \}_{j=1}^M, q_j \in \mathbb{R}$.
    Verjetnost za prehod je tedaj sorazmerna:

    \begin{equation}\label{eq1}
        P_{q_1, q_2} \propto \exp \left( -\frac{M}{2\beta}(q_2 - q_1)^2 - \frac{\beta}{M}V(q_1) \right)
    \end{equation}

    Metropolisov algoritem je naslednji:

    \begin{enumerate}
        \item Izberemo si začetno konfiguracijo - elemente $\ket{q_j}$ generiramo po normalni porazdelitvi.
        \item Izberemo si naključen element $\ket{q_j}$.
        \item Predlagamo potezo $q'_j = q_j + \xi$, kjer je $\xi \in (-\varepsilon, \varepsilon)$ enakomerno porazdeljen.
        \item Potezo sprejmemo z verjetnostjo:
        \begin{equation}\label{eq2}
            min \left\{ 1, \frac{P_{q_{j-1}, q'_j} P_{q'_j, q_{j+1}}}{P_{q_{j-1}, q_j} P_{q_j, q_{j+1}}} \right\},
        \end{equation}
        \begin{equation}\label{eq3}
            \begin{split}
                \frac{P_{q_{j-1}, q'_j} P_{q'_j, q_{j+1}}}{P_{q_{j-1}, q_j} P_{q_j, q_{j+1}}} =
                \exp \biggl[ &-\frac{M}{\beta}\left(q_j^{\prime 2} - q_j^2 + (q_j - q'_j)(q_{j-1} + q_{j+1})\right) \\
                            &- \frac{\beta}{M}\left(V(q'_j) - V(q_j)\right) \biggr]
            \end{split}
        \end{equation}
    \end{enumerate}

    Raziskali bomo povprečne vrednosti celotne, kinetične in potencialne energije v odvisnosti od inverzne temperature
    $\beta$.
    Po znani termodinamski identiteti pridemo do izraza za povprečno celotno energijo:

    \begin{equation}\label{eq5}
        \expected{H} = \expected{\frac{M}{2\beta} - \frac{M}{2\beta^2}\sum_{j=1}^M (q_{j+1} - q_j)^2
                                + \frac{1}{M}\sum_{j=1}^M V(q_j)}
    \end{equation}

    Za izračun povprečne vrednosti kinetične energije $\expected{T}$ z Metropolisovim algoritmom vzorčimo prva dva člena
    na desni strani zgornjega izraza, za izračun potencialne $\expected{V}$ pa tretjega.
    Povprečna vrednost celotne energije je tedaj vsota $\expected{H} = \expected{T} + \expected{V}$.

    \section{Harmonski oscilator}

    Ker poznamo lastne energije $E_n = n + \half$, lahko pričakovan rezultat izrazimo povsem analitično:

    \begin{equation}\label{eq6}
        \expected{H} = -\pder{\log{z(\beta)}}{\beta}, \quad
        z(\beta) = \sum_n e^{-\beta E_n} = \frac{1}{2 \sinh{\frac{\beta}{2}}}
    \end{equation}
    \begin{equation}\label{eq7}
        \Rightarrow \expected{H} = \half \coth{\frac{\beta}{2}}
    \end{equation}

    Virialni teorem nam pove, da sta povprečni kinetična in potencialna energija enaki:

    \begin{equation}\label{eq8}
        \expected{T} = \expected{V} = \frac{\expected{H}}{2}
    \end{equation}

    Uporabimo sedaj Monte-Carlo metodo.
    Simulacija naj teče od nizkih $\beta$ proti visokim.
    Pred začetkom je izvedenih $100 \times M^2$ poskusov Metropolisovih potez, da se vzpostavi smiselno začetno stanje.
    Pri vsaki novi temperaturi je nato pred začetkom narejenih še $M^2$ potez, da se sistem uravnovesi.
    Nato steče $10000 \times M$ potez, pri čemer na vsakih $128$ odčitamo kinetično in potencialno energijo. \\

    Poglejmo si, kaj pokaže naša simulacija: slika~\ref{slika1}a prikazuje popolno ujemanje s teorijo v limiti nizkih
    temperatur ($\beta \to \infty$), a zelo slabo ujemanje pri višjih temperaturah ($\beta \to 0$).
    Še huje je, kar vidimo na sliki~\ref{slika1}b: metoda odpove tudi pri prenizkih temperaturah (visokih $\beta$).
    To odstopanje je preprosta posledica dejstva, da so prehodne verjetnosti v naši metodi izražene zgolj do prvega
    reda v $\frac{\beta}{M}$.

    \begin{figure}
        \centering
        \begin{subfigure}{\textwidth}
            \includegraphics[width = \textwidth]{slika1.pdf}
            \caption{}
        \end{subfigure}
        \begin{subfigure}{\textwidth}
            \includegraphics[width = \textwidth]{slika2.pdf}
            \caption{}
        \end{subfigure}
        \caption{Rezultati simulacije termalnega stanja harmonskega oscilatorja.
        S $H_a$ je označena analitično izračunana vrednost povprečne energije.}
        \label{slika1}
    \end{figure}

    \section{Anharmonski oscilator}

    \ldots

    \begin{figure}
        \centering
        \begin{subfigure}{\textwidth}
            \includegraphics[width = \textwidth]{slika3.pdf}
            \caption{}
        \end{subfigure}
        \begin{subfigure}{\textwidth}
            \includegraphics[width = \textwidth]{slika4.pdf}
            \caption{}
        \end{subfigure}
        \caption{Rezultati simulacije termalnega stanja anharmonskega oscilatorja.
        S $H_a$ je označena analitično izračunana vrednost povprečne energije.}
        \label{slika2}
    \end{figure}

\end{document}
