\documentclass[a4paper]{article}
\usepackage[slovene]{babel}
\usepackage[utf8]{inputenc}
\usepackage[T1]{fontenc}
%\usepackage[margin=2cm, bottom=3cm, foot=1.5cm]{geometry}
\usepackage{float}
\usepackage{graphicx}
\usepackage{amsmath}
\usepackage{amssymb}
\usepackage{subcaption}
\usepackage{hyperref}

\newcommand{\tht}{\theta}
\newcommand{\Tht}{\Theta}
\newcommand{\dlt}{\delta}
\newcommand{\eps}{\epsilon}
\newcommand{\thalf}{\frac{3}{2}}
\newcommand{\ddx}[1]{\frac{d^2#1}{dx^2}}
\newcommand{\ddr}[2]{\frac{\partial^2#1}{\partial#2^2}}
\newcommand{\mddr}[3]{\frac{\partial^2#1}{\partial#2\partial#3}}

\newcommand{\der}[2]{\frac{d#1}{d#2}}
\newcommand{\pder}[2]{\frac{\partial#1}{\partial#2}}
\newcommand{\half}{\frac{1}{2}}
\newcommand{\forth}{\frac{1}{4}}
\newcommand{\q}{\underline{q}}
\newcommand{\p}{\underline{p}}
\newcommand{\x}{\underline{x}}
\newcommand{\liu}{\hat{\mathcal{L}}}
\newcommand{\bigO}[1]{\mathcal{O}\left( #1 \right)}
\newcommand{\pauli}{\mathbf{\sigma}}
\newcommand{\bra}[1]{\langle#1|}
\newcommand{\ket}[1]{|#1\rangle}
\newcommand{\id}[1]{\mathbf{1}_{2^{#1}}}
\newcommand{\tinv}{\frac{1}{\tau}}
\newcommand{\s}{\sigma}
\newcommand{\us}{\underline{\s}}
\newcommand{\vs}{\vec{\s}}
\newcommand{\vr}{\vec{r}}
\newcommand{\vq}{\vec{q}}
\newcommand{\vv}{\vec{v}}
\newcommand{\vo}{\vec{\omega}}
\newcommand{\uvs}{\underline{\vs}}
\newcommand{\expected}[1]{\langle #1 \rangle}
\newcommand{\D}{\Delta}

\begin{document}

    \title{\sc\large Višje računske metode\\
		\bigskip
		\bf\Large Kvantni Monte-Carlo}
	\author{Mitja Vodnik, 28182041}
	\date{\today}
	\maketitle

    S pomočjo Metropolisovega algoritma bomo raziskali ravnovesna termalna stanja klasičnih Isingovih in Heisenbergovih
    spinov.

    \begin{equation}\label{eq1}
        \expected{a} = \int d^N\x w(\x) a(\x) = \lim_{M \to \infty} \frac{1}{M} \sum_{j=1}^M a(\x_j)
    \end{equation}

    \section{2D Isingov model}

    \iffalse
    \begin{figure}
        \centering
        \begin{subfigure}{\textwidth}
            \includegraphics[width = \textwidth]{slika1.pdf}
            \caption{}
        \end{subfigure}
        \begin{subfigure}{\textwidth}
            \includegraphics[width = \textwidth]{slika2.pdf}
            \caption{}
        \end{subfigure}
        \begin{subfigure}{\textwidth}
            \includegraphics[width = \textwidth]{slika3.pdf}
            \caption{}
        \end{subfigure}
        \caption{Fazni prehodi Isingovega modela za različno velike mreže.
        Oranžna črta prilagojena na graf magnetizacije predstavlja Onsagerjevo formulo za spontano magnetizacijo.
        Z rdečo vertikalno črto je označena kritična temperatura določena s to prilagoditvijo.
        Vse je računano pri sklopitveni konstanti $J = 1$.}
        \label{slika1}
    \end{figure}
    \fi

\end{document}
